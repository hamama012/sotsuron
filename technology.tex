\chapter{利用技術}
本章では,本研究に利用する各技術について述べる.

\section{Unity}

Unityとは,ユニティ・テクノロジーズが開発した総合開発環境を内蔵するゲームエンジンである.
モバイル,VR,AR,デスクトップ,コンソール,Webと数多くのプラットホームに対応しており,その柔軟性は極めて高い.簡易なものであればプログラミングを行わずとも3Dステージの構築や物理エンジンの導入が可能である.また3Dだけでなく,2D描画にも対応している.
更に,アセットストアと呼ばれるショップで他のユーザが作成したプログラムや3DCGモデル,テクスチャといったデータがアセットという形で提供されており,プロジェクトに直接インポートして使用することができる.
3D空間の作成が容易であることから,本研究ではUnityを使用する.

\begin{figure}[htb]
\centering
\includegraphics*[width=16cm,clip]{images/sc4.eps}
\caption{Unityの開発画面}
\label{fig:unity}
\end{figure}

\section{HTC Vive}

HTC Viveとは,HTCとValve Corporationの共同で開発されたバーチャルリアリティ向けヘッドマウントディスプレイである.
部屋の二箇所にベースステーションを設置することで,最小1.5m×2m,最大3m×4mの範囲でユーザの動きをリアルタイムにトラッキング,仮想現実に反映できる.VR専用に設計されたコントローラが付属しており,直感的な操作とジェスチャーを実現可能.
本研究では現実感を高めるためにHTC Viveを,接触フィードバックの実装にコントローラを使用する.

\begin{figure}[htb]
\centering
\begin{tabular}{c}

\begin{minipage}{0.50\hsize}
\includegraphics*[width=7.8cm,clip]{images/technology01.eps}
\caption{HTC Vive}
\label{fig:htc}
\end{minipage}

\begin{minipage}{0.50\hsize}
\includegraphics*[width=7.8cm,clip]{images/technology02.eps}
\caption{コントローラ}
\label{fig:controller}
\end{minipage}

\end{tabular}
\end{figure} 

\section{Maya}

Mayaはオートデスク社によって開発されたハイエンド3次元コンピュータグラフィックスソフトウェアである.
ハリウッドをはじめとした映画からゲーム,CMの制作に使用されている.映画のVFX市場で圧倒的なシェアを誇り,ゲーム市場においてもMayaと同じく3次元コンピュータグラフィックスソフトウェアである3ds Maxと50%50%であるとオートデスク社は述べている.
本研究で使用したペット及びユーザの手のアバターはMayaを用いて作成した.

\begin{figure}[htb]
\centering
\includegraphics*[width=16cm,clip]{images/sc3.eps}
\caption{Mayaの制作画面}
\label{fig:maya}
\end{figure}

\section{Visual Studio}

Visual StudioはMicrosoft社が開発した統合開発環境である.C#,Visual Basic,Pythonといった8種類のプログラム言語に対応している.また,WindowsだけでなくiOSやLinuxなどのOSにも対応可能である.
UnityにはMonoDevelopという統合開発環境が標準で備わっているが,Visual Studioがこのスクリプトエディタよりも高機能であることから,本研究ではVisual Studioを利用する.