%付録
%ソースコードを付ける場合は,verbatim 環境で囲む.
%verbatim 環境は場合によっては,自動階行されないかもしれない.
%改行されないときは,自分で適度なところで区切ってください.

\appendix
\chapter{コルーチン処理によるUnity-chanの発生}

\begin{verbatim}
using UnityEngine;
using System.Collections;

public class hassei: MonoBehaviour
{

    public GameObject cube;

    void Start()
    {
        StartCoroutine("Sample");
    }

    IEnumerator Sample()
    {
        yield return new WaitForSeconds(1.0f);
        for (int i = 1; i < 10; i++)
        {
            yield return new WaitForSeconds(1.0f);
            if (i == 3)
            {
                float x = Random.Range(0.0f, 1.0f);
                float y = Random.Range(0.0f, 0.0f);
                float z = Random.Range(0.0f, 50.0f);
                Instantiate(cube, new Vector3(x, y, z), Quaternion.identity);
            }
        }
    }
    void OnTriggerEnter(Collider other)
    {
        Destroy(gameObject);
    }
}
\end{verbatim}

\chapter{NavMeshに従い目的地に移動する}

\begin{verbatim}
using UnityEngine;
using System.Collections;

public class hassei: MonoBehaviour
{

    public GameObject cube;

    void Start()
    {
        StartCoroutine("Sample");
    }

    IEnumerator Sample()
    {
        yield return new WaitForSeconds(1.0f);
        for (int i = 1; i < 10; i++)
        {
            yield return new WaitForSeconds(1.0f);
            if (i == 3)
            {
                float x = Random.Range(0.0f, 1.0f);
                float y = Random.Range(0.0f, 0.0f);
                float z = Random.Range(0.0f, 50.0f);
                Instantiate(cube, new Vector3(x, y, z), Quaternion.identity);
            }
        }
    }
    void OnTriggerEnter(Collider other)
    {
        Destroy(gameObject);
    }
}
\end{verbatim}

