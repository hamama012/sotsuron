\chapter{結論}
本章では,本研究における関連研究を述べる.

\section{考察}

バーチャルペットが与える現実感として、最も大きな影響を与えるのはペットの動作部分であると考える。


また、これらの動作にはペットの3Dモデルも深く関わっていると考える。本研究では現実に存在している犬をモデルにし、出来る限り本物に近い造形を目指しモデリングを行った。3Dモデル自体は実験中の被験者から可愛いといった声があがるなど好評であったが、その本物に近い3Dモデルの動作が被験者に違和感を生じさせた。
例えばペットが顔の向きを変える動作では、接触して一定時間経過すると視線先ターゲットが手からユーザの目部分へと変更される。その際、ペットのアニメーションが待機状態へと戻り動きがなくなってしまう。一切動かずユーザを見つめる状態になってしまい、怖かったという声が複数あった。
すなわち、動作における不気味の谷現象が起きていたと予想される。
3Dモデルのリアリティを追求するのであれば、そのリアリティが高ければ高いほど精密な動きが求められる。これは制作側にとって非常に負担がかかることが想定できる。
現実感の向上によるバーチャルペットのセラピー効果増幅を狙ったものであり、リアリティのみの追求は本意ではない。
3Dモデルとその動作についてはデフォルメ化し違和感を減らすか、あるいは既存の動物と全く異なるバーチャルペットを新たに作り出す

\section{まとめ}

本研究で開発したバーチャルペットに身体性を付与することができた。


\section{今後の課題}