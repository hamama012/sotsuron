\chapter{序論}
本章では,本研究における背景と目的を中心として述べる.

\section{背景}

近年,精神疾患を抱える患者が増加している.その要因として,人間関係や環境変化・身体の不調等から来るストレスがある.
ストレスはうつ病といった精神疾患のみならず,神経性胃炎・十二指腸潰瘍を引き起こすなど身体に悪影響を及ぼす.ストレス社会とも呼ばれる現代を生きる上でメンタルヘルス対策への取り組みは必要不可欠といえる.

ストレス対処法の1つとして,アニマルセラピー(Animal Assisted Therapy)がある.動物と触れ合うことにより精神・身体機能の向上を図る治療法で,実際にリラックス効果や社会性の向上といった精神への効果,身体面においても血圧の低下などの効果が確認されている.
しかし,この治療法には生きた動物と接するが故のリスクが存在する.
例えば不特定多数の人間が集まる施設では,セラピーを受ける患者以外のアレルギーを持つ人や動物に苦手意識がある人への配慮が必要である.
更に,セラピーに使用する動物には吠える・噛むといった問題行動をしないよう徹底的な訓練を行わなくてはならず,また動物を管理するため医師や看護師とは別に専門的な訓練を受けた人間が必要になる.

アニマルセラピーの代替案として,近年ペットロボットやバーチャルペットの研究が進められている.
ペットロボットとは,実用性よりも娯楽性に重きを置いて設計されたロボットのことである.動きなど目で楽しませるものから,対話を行うことができるものまでその機能は幅広い.当初はペットロボットという名前通り動物の形を模したものが多く存在したが,現在は人の形をしたロボットも開発されている.
中でも産業技術総合研究所が開発したアザラシ型ロボット「パロ」は2002年に世界一の癒しロボットとしてギネスブックにより認定されており,現在においても医療機関・介護施設等で活躍している.2017年にはSonyによって「aibo」が発表されるなどその発展は目覚ましい.

一方,バーチャルペットはゲームソフトやスマートフォンアプリなど,気軽に触れることのできるコンテンツが多く存在する.最も知名度の高いものとして,たまごっちがあげられる.画面上に存在する「たまごっち」と呼ばれる架空の存在を飼育し,成長させることを目的としたキーチェーンゲームで,1990年代において社会現象を巻き起こした.現代ではステージである家の庭先に猫を集めるスマートフォンアプリ「ねこあつめ」が人気を博し,2015年にはCEDEC AWARDS2015 ゲームデザイン部門最優秀賞を受賞している.
%たまごっち,ねこあつめの画像いれる?

\section{関連研究}
林らはペットロボットとバーチャルペットとのふれあいによるセラピー効果の差異を身体性の観点から比較検証を行った.
身体性とは身体が持つ性質を指すが,その定義は分野ごとに変わる.林らは実体を有することにより,ユーザや周辺環境との物理的な相互作用を可能とすることを身体性として定義している.
実験の結果,バーチャルペットに比べてペットロボットの方が心理的・生理的共にストレス緩和効果が高いことが分かった.その要因として,ペットロボットが身体性を有することを述べている.\\
 しかしながら,バーチャルペットにおいてもVR技術を用いることにより,疑似的ではあるが身体性を持つことは可能であると考える.
VR(Virtual Reality)とは,人間の五感を刺激することでまるで現実のように体感させる概念や技術のことであり,広義で言えば映画やTVなどもこれに含まれる.本研究で用いるVR技術は,現在主流となっている頭部に装着するヘッドマウントディスプレイ(HMD)を用いて360度の視界を表現し,コンピュータで生成された空間にまるで現実のように没入することができるシステムを指す.
身体性がセラピー効果と深い繋がりを持つことから,このような技術を用いて身体性を有することができればバーチャルペットのセラピー効果向上が期待できる.

%身体性についてはwikipediaにも記載..「認知科学,人工知能の分野では,物理的な身体があることによって,環境との相互作用ができることにより,学習や知能の構築にもたらす効果や性質を指す」

\section{目的}
そこで本研究では,VR技術を利用して身体性の特性である現実感を向上させ,更に接触フィードバックを実装することによって身体性を有するバーチャルペットの開発を目的とする.


\section{本論文の構成}
本論文は,本章を含め6章から構成される.第1章では,序論として背景,目的と,本論文の構成について述べた.第2章では,先行研究として,ペットロボットとバーチャルペットの比較検証実験について述べる.第3章では,利用技術としてHTC ViveやUnity等について述べる.第4章では,本研究で開発するバーチャルペットの設計について述べる.第5章では,実験環境,実験方法,実験結果について述べる.第6章では,まとめと今後の課題について述べる.